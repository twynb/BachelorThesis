\chapter{Interpolating Intersection Checks}

To bounce a ray through a scene, checks need to be performed to know which object the ray intersects with first and thus bounces off of.
These checks are usually done by modelling objects using an equation that describes whether a point is on the object surface or not,
then injecting the ray's function into that equation and resolving.
With static objects, this check is trivial:
The only parameter to resolve for is the intersection time \(t\), and said parameter only occurs in the ray function.
\newline
With moving objects, this check becomes more complicated: The equation still needs to be resolved for \(t\),
but now \(t\) is also part of the surface equation itself.
In this chapter, equations are established and solved to perform intersection checks with moving spheres and surfaces,
as those two should already be sufficient to simulate most scenes.
Resolution for other primitives, such as quadric surfaces, should be derivable in a similar way.
\newline
A ray \(R\)'s position at time \(t\), with \(R\) starting at the point \(P\) at time \(t_0\) and travelling in the direction \(v\) can be defined as follows:

\begin{equation}\label{RayEq}
    R(t) = P + (t - t_0) \cdot v
\end{equation}

Note that unlike with most ray tracing implementations, the magnitude of the direction vector \(v\) is significant:
Since the movement of objects checked for intersections is also dependent on time,
objects and the ray must move at the same time scale.
\(v\) must accordingly be scaled to fit the velocity of the ray and the implementation's scale for \(t\) and coordinates.
Otherwise, the intersection calculation will yield incorrect results as at the resulting intersection times,
the ray will either not have travelled to the resulting intersection point yet or will have already surpassed it.

\section{Intersection Checks for Spheres}

A sphere \(C\) is modelled as having a radius \(r\) and a collection of keyframes \(k_n\), each with a center point \(c_{k_n}\) and a time \(t_{k_n}\).
Intersection calculations need to be run separately for each set of consecutive keyframes \(k_1\) and \(k_2\).
\newline

The center point \(c_t\) of \(C\) at time \(t\) can be defined as

\begin{equation}
    c_t = m \cdot c_{k_1} + (1-m) \cdot c_{k_2}
\end{equation}

With \(m\) representing the interpolation point between the two keyframes.
For the scope of this thesis, linear interpolation is used, leading to \(m\) being defined as

\begin{equation}\label{MDef}
    m = \frac{t_{k_2} - t}{t_{k_2} - t_{k_1}}
\end{equation}

For non-linear interpolation modes between two keyframes (such as using a sinusoidal function),
\(m\) is defined differently. The math below is still the same until \eqref{SphereBeforeM} where \(m\) is resolved,
at which point the new definition of \(m\) must be substituted instead of the definition above.
Resolving from there should be trivial, depending on the definition of \(m\).
\newline
For interpolation modes that work with more than two keyframes (such as splines),
\(c_t\) would instead need to be defined using that other interpolation function,
and the scope of which set of keyframes applies to which range of \(t\) would need to be limited accordingly.
Exploring this is outside the scope of this thesis.
\newline

For any given time \(t\), the surface of \(C\) is then defined as all points \(x\) for which

\begin{equation}\label{SphereDef}
    \|x-c_t\|^2 = r^2
\end{equation}

Inserting \eqref{RayEq} in place of \(x\) in \eqref{SphereDef}:

\begin{equation}
    \|P + (t - t_0) \cdot v - c_t\|^2 = r^2
\end{equation}

\begin{equation*}
    (P - c_{k_2} + t \cdot v - t_0 \cdot v + m \cdot (c_{k_2} - c_{k_1})) \cdot (P - c_{k_2} + t \cdot v - t_0 \cdot v + m \cdot (c_{k_2} - c_{k_1})) = r^2
\end{equation*}

Resolving this and defining \(\delta c = c_{k_2} - c_{k_1}\) results in a function of \(t\) and \(m\):

\begin{equation}\label{SphereBeforeM}
    \begin{split}
        \|P - c_{k_2}\|^2
        - 2 \cdot t_0 \cdot P \cdot v
        + 2 \cdot t_0 \cdot c_{k_2} \cdot v
        + t_0^2 \cdot \|v\|^2
        \\
        + t \cdot 2 \cdot (P \cdot v - c_{k_2} \cdot v)
        - t \cdot 2 \cdot t_0 \cdot \|v\|^2
        \\
        + m \cdot 2 \cdot (P \cdot dc - c_{k_2} \cdot \delta c)
        - m \cdot 2 \cdot t_0 \cdot v \cdot \delta c
        \\
        + t \cdot m \cdot 2 \cdot v \cdot \delta c
        + t^2 \cdot \|v\|^2
        + m^2 \cdot \|\delta c\|^2
        \\
        = r^2
    \end{split}
\end{equation}

Now, to get a polynomial function of \(t\), \(m\) needs to be resolved.
With the linear definition of \(m\) from above, this results in the following equation:

\begin{equation*}
    \begin{split}
        \|P - c_{k_2}\|^2
        - 2 \cdot t_0 \cdot P \cdot v
        + 2 \cdot t_0 \cdot c_{k_2} \cdot v
        + t_0^2 \cdot \|v\|^2
        \\
        + t \cdot 2 \cdot (P \cdot v - c_{k_2} \cdot v)
        - t \cdot 2 \cdot t_0 \cdot \|v\|^2
        \\
        + \frac{(t_{k_2} - t) \cdot
            2 \cdot (P \cdot \delta c - c_{k_2} \cdot \delta c - t_0 \cdot v \cdot \delta c)
        }{t_{k_2} - t_{k_1}}
        + \frac{t \cdot (t_{k_2} - t) \cdot 2 \cdot v \cdot \delta c}{t_{k_2} - t_{k_1}}
        \\
        + t^2 \cdot \|v\|^2
        + \frac{(t_{k_2} - t)^2 \cdot \|\delta c\|^2}{(t_{k_2} - t_{k_1})^2}
        \\
        = r^2
    \end{split}
\end{equation*}

Introducing \(\delta t = t_{k_1} - t_{k_2}\) for readability, this resolves to a 2nd degree polynomial:

\begin{equation}
    d_2t^2 + d_1t + t_0 = 0
\end{equation}

with

\begin{equation}
    d_2 = \|v\|^2 \cdot \delta t^2
    + \|\delta c\|^2
    - 2 \cdot v \cdot \delta c \cdot \delta t
\end{equation}
\begin{equation}
    d_1 = 2 \cdot (
    (P - c_{k_2}) \cdot v \cdot \delta t^2
    - t_0 \cdot \|v\|^2
    - (P - c_{k_2} - t_0 \cdot v) \cdot \delta c \cdot \delta t
    + t_{k_2} \cdot v \cdot \delta c \cdot \delta t
    - t_{k_2} \cdot \|\delta c\|^2
    )
\end{equation}
\begin{equation}
    \begin{split}
        d_0 = (
        \|P - c_{k_2}\|^2
        + 2 \cdot t_0 \cdot (c_{k_2} - P) \cdot v
        + t_0^2 \cdot \|v\|^2
        ) \cdot \delta t^2
        \\
        + t_{k_2} \cdot 2 \cdot (P - c_{k_2} - t_0 \cdot v) \cdot dc \cdot \delta t
        + t_{k_2}^2 \cdot \|dc\|^2
        - r^2 \cdot \delta t^2
    \end{split}
\end{equation}

If \(r\) can also be varied between keyframes,
replace \(r^2\) with the according interpolated value \((m - r_{k_1} + (1-m) r_{k_2})^2\) and resolve accordingly.
\newline
The real roots of this equation represent all times at which the sphere and ray would theoretically intersect.
Note that all roots \(t\) where \(t < t_{k_1}\) or \(t > t_{k_2}\) must be discarded
because the surface is not described by \(k_1\) and \(k_2\) outside the time frame between them.
Additionally, all roots where \(t \le t_0\) must be discarded as these intersections would happen before the ray started,
behind its starting location.
\newline
Of the real roots remaining after this filter, the lowest result for \(t\) is the one where the ray and sphere intersect,
assuming the ray does not bounce off of a different object before that.
The coordinates at which the intersection takes place can be calculated by calculating \(R(t)\).

% TODO find case where d2=d1=d0=0

\section{Intersection Checks for Surfaces}

Similarly to spheres, a polygonal surface \(S\) can also be modelled using a set of keyframes \(k_n\),
each with points \(P_{1..o, k_n}\) for the corners of the surface, with \(o \ge 3\) and \(o\) being consistent between keyframes.
As with spheres, intersection calculations need to be run separately for each set of consecutive keyframes \(k_1\) and \(k_2\).
\newline
Using \(m\) from \eqref{MDef} with the same caveats, each point \(P_{n, t}\) is calculated as

\begin{equation}\label{SurfacePointDef}
    P_{n, t} = m \cdot P_{n, k_1} + (1 - m) \cdot P_{n, k_2}, 1 \le n \le o
\end{equation}

Assuming all points of the polygon are within one surface,
the surface containing the polygon at any given time \(t\) can be described as the set of all points \(x\) where

\begin{equation}\label{SurfaceDef}
    (x - P_{1, t}) \cdot ((P_{2, t} - P_{1, t}) \times (P_{3, t} - P_{1, t})) = 0
\end{equation}

Injecting \eqref{RayEq} into \eqref{SurfaceDef}:

\begin{equation}
    (P + t \cdot v - t_0 \cdot v - P_{1, t}) \cdot ((P_{2, t} - P_{1, t}) \times (P_{3, t} - P_{1, t})) = 0
\end{equation}

Exploiting that the vector cross product is distributive over addition (i.e. \( (x + y) \times z = x \times z + y \times z\)):

\begin{equation*}
    (P + t \cdot v - t_0 \cdot v - P_{1, t}) \cdot
    (P_{2, t} \times P_{3, t} - P_{2, t} \times P_{1, t} - P_{1, t} \times P_{3, t} + P_{1, t} \times P_{1, t})
\end{equation*}

As the cross product of a vector with itself is always 0, the last part can be discarded:

\begin{equation}\label{SurfaceBeforeCross}
    (P + t \cdot v - t_0 \cdot v - P_{1, t}) \cdot
    (P_{2, t} \times P_{3, t} - P_{2, t} \times P_{1, t} - P_{1, t} \times P_{3, t})
\end{equation}

Each of these cross products should be solved individually before resolving the full equation.
Using \eqref{SurfacePointDef}:

\begin{equation}
    P_{a, t} \times P_{b, t}
    = ((1-m) \cdot P_{a, k_2} + m \cdot P_{a, k_1}) \times ((1-m) \cdot P_{b, k_2} + m \cdot P_{b, k_1})
\end{equation}

Again exploiting that the cross product is distributive over addition, as well as the fact that with a scalar \(a\),
\(a \cdot (x \times y) = x \times (a \cdot y)\), this can be resolved to a polynomial function of \(m\):

\begin{equation}
    \begin{split}
        P_{a, t} \times P_{b, t} =
        P_{a, k_2} \times P_{b, k_2}
        + m \cdot (
        - 2 (P_{a, k_2} \times P_{b, k_2})
        + (P_{a, k_1} \times P_{b, k_2})
        + (P_{a, k_2} \times P_{b, k_1})
        )
        \\
        + m^2 \cdot (
        (P_{a, k_2} \times P_{b, k_2})
        - (P_{a, k_1} \times P_{b, k_2})
        - (P_{a, k_2} \times P_{b, k_1})
        + (P_{a, k_1} \times P_{b, k_1})
        )
    \end{split}
\end{equation}

As with \eqref{SphereBeforeM}, to get a function of \(t\),
\(m\) needs to replaced with its definition (\eqref{MDef} in this case):

\begin{multline*}
    P_{a, t} \times P_{b, t} =
    \\
    P_{a, k_2} \times P_{b, k_2}
    + \frac{(t_{k_2} - t) \cdot (
        - 2 (P_{a, k_2} \times P_{b, k_2})
        + (P_{a, k_1} \times P_{b, k_2})
        + (P_{a, k_2} \times P_{b, k_1})
        )}{t_{k_2} - t_{k_1}}
    \\
    + \frac{(t_{k_2} - t)^2 \cdot (
        (P_{a, k_2} \times P_{b, k_2})
        - (P_{a, k_1} \times P_{b, k_2})
        - (P_{a, k_2} \times P_{b, k_1})
        + (P_{a, k_1} \times P_{b, k_1})
        )}{(t_{k_2} - t_{k_1})^2}
\end{multline*}

Again introducing \(\delta t = t_{k_2} - t_{k_1}\) for readability,
this can be resolved to a second-degree polynomial:

\begin{equation}\label{SurfaceCrossPoly}
    P_{a, t} \times P_{b, t} = f_{2, a, b}t^2 + f_{1, a, b}t + f_{0, a, b}
\end{equation}

with

\begin{equation}
    f_{2, a, b} = (P_{a, k_2} \times P_{b, k_2})
    - (P_{a, k_1} \times P_{b, k_2})
    - (P_{a, k_2} \times P_{b, k_1})
    + (P_{a, k_1} \times P_{b, k_1})
\end{equation}
\begin{equation}
    \begin{split}
        f_{1, a, b} = - \delta t \cdot (
        - 2 (P_{a, k_2} \times P_{b, k_2})
        + (P_{a, k_1} \times P_{b, k_2})
        + (P_{a, k_2} \times P_{b, k_1})
        )
        \\
        - 2 \cdot t_{k_2} \cdot (
        (P_{a, k_2} \times P_{b, k_2})
        - (P_{a, k_1} \times P_{b, k_2})
        - (P_{a, k_2} \times P_{b, k_1})
        + (P_{a, k_1} \times P_{b, k_1})
        )
    \end{split}
\end{equation}
\begin{equation}
    \begin{split}
        f_{0, a, b} = \delta t^2 \cdot (P_{a, k_2} \times P_{b, k_2})
        \\
        + t_{k_2} \cdot \delta t \cdot (
        - 2 (P_{a, k_2} \times P_{b, k_2})
        + (P_{a, k_1} \times P_{b, k_2})
        + (P_{a, k_2} \times P_{b, k_1})
        )
        \\
        + t_{k_2}^2 \cdot (
        (P_{a, k_2} \times P_{b, k_2})
        - (P_{a, k_1} \times P_{b, k_2})
        - (P_{a, k_2} \times P_{b, k_1})
        + (P_{a, k_1} \times P_{b, k_1})
        )
    \end{split}
\end{equation}

Inserting \eqref{SurfaceCrossPoly} into \eqref{SurfaceBeforeCross}:

\begin{equation*}
    (P + t \cdot v - t_0 \cdot v - P_{1, t}) \cdot
    (f_{2, 2, 3} t^2 + f_{1, 2, 3} t + f_{0, 2, 3} - f_{2, 2, 1} t^2 - f_{1, 2, 1} t - f_{0, 2, 1} - f_{2, 1, 3} t^2 - f_{1, 1, 3} t - f_{0, 1, 3})
    = 0
\end{equation*}

Introducing \(g_n = f_{n, 2, 3} - f_{n, 2, 1} - f_{n, 1, 3}\) for readability,
resolving \(P_{1, t}\) as per \eqref{SurfacePointDef} and introducing \(\delta P_1 = P_{1, k_2} - P_{1, k-1}\):

\begin{equation}
    (P + t \cdot v - t_0 \cdot v - P_{1, k_2} + \frac{t_{k_2} \delta P_1}{\delta t} - \frac{t \cdot \delta P_1}{\delta t}) \cdot
    (
    t^2 g_2
    + t g_1
    + g_0
    )
    = 0
\end{equation}

This can be resolved to a third degree polynomial:

\begin{equation}
    t^3d_3 + t^2d_2 + td_1 + d_0 = 0
\end{equation}

With

\begin{equation}
    d_3 = g_2 \cdot v
    - \frac{g_2 \cdot \delta P_1}{\delta t}
\end{equation}
\begin{equation}
    d_2 = g_2 \cdot P
    - t_0 \cdot g_2 \cdot v
    - g_2 \cdot P_{1, k_2}
    + \frac{t_{k_2} \cdot g_2 \cdot \delta P_1}{\delta t}
    + g_1 \cdot v
    - \frac{g_1 \cdot \delta P_1}{\delta t}
\end{equation}
\begin{equation}
    d_1 = g_1 \cdot P
    - t_0 \cdot g_1 \cdot v
    - g_1 \cdot P_{1, k_2}
    + \frac{t_{k_2} \cdot g_1 \cdot \delta P_1}{\delta t}
    + g_0 \cdot v
    - \frac{g_0 \cdot \delta P_1}{\delta t}
\end{equation}
\begin{equation}
    d_0 = g_0 \cdot P
    - t_0 \cdot g_0 \cdot v
    - g_0 \cdot P_{1, k_2}
    + \frac{t_{k_2} \cdot g_0 \cdot \delta P_1}{\delta t}
\end{equation}

This polynomial can then be solved using a general cubic formula such as the one described by Abramowitz and Stegun~\cite{AS48}
or Flocke's Algorithm~\cite{Fl15}.
\newline
As with spheres, the real roots of this equation are all the points in time at which the ray and the surface would meet.
Roots \(t\) where \(t < t_{k_1}\) or \(t > t_{k_2}\), as well as roots where \(t \le t_0\),
must be discarded as explained above.
\newline
For the remaining roots, another check needs to be done
for whether the intersection point \(R(t)\) is actually inside the polygon, and not just on the same surface.
With triangles, this can be trivially done by calculating the homogenous barycentric coordinates \(\alpha, \beta, \gamma\) of \(R(t)\)
and verifying that \(0 \le \alpha \le 1, 0 \le \beta \le 1, 0 \le \gamma \le 1, \alpha + \beta + \gamma = 1\).
\newline
The root with the lowest value that satisfies the above conditions again represents the time where the ray and polygon intersect,
assuming there is no intersection happening before. If no root satisfies these conditions, no intersection happens.