\chapter{Optimisation through spatial/temporal division}

As Whitted already noted in 1980~\cite{Wh80}, even in a ray tracing system with static checks,
intersection calculations take up the vast majority of processing time (between 75-95\% in Whitted's case).
Since this effect will only increase with more expensive intersection checks,
the amount of checks run per ray should be reduced as much as possible.
A method for this will be evaluated in this chapter.

\section{Chunks vs. Bounding Volumes}

One common optimisation for ray tracing systems is to limit the amount of intersection calculations by eliminating
objects the ray cannot intersect with in a simpler way.
There are two general sets of methods used for this:
\newline
Bounding Volume Hierarchies (BVHs), first proposed by Clarke~\cite{Cl76},
work by enclosing each object in the scene within a volume containing it.
This bounding volume uses a simpler geometric primitive that allows for faster intersection checks than the object itself,
usually quadric surfaces or spheres.
These bounding volumes are then grouped into bigger bounding volumes, forming a hierarchical tree structure.
Rays then walk down the tree structure, checking for intersections with the corresponding bounding volumes.
If it does not intersect with a branch's bounding volume,
any objects within that branch can be ignored for further intersection checks.
\newline
Another method first proposed as a Three Dimensional Digital Differential Analyzer by Fujimoto and Iwata~\cite{FI85},
instead divides a scene into separate cells (chunks),
with each chunk keeping a list of which objects are inside it.
Rays can then traverse from chunk to chunk along their trajectory
and only check for intersections with the objects contained in the chunk they're currently in.
\newline
Since objects can move around the scene, using one of these methods without changes becomes inefficient.
If, for example, a receiver moves from one end of the scene to the other over the course of ten seconds,
its bounding volume would extend over all of that distance for the entirety for the scene, despite it not touching
the majority of it for the most part. Similarly, it would be kept in its starting position's chunk for the entirety
of the scene despite leaving that area very early, making for needless intersection checks.
\newline
For this use case, chunks become a lot more efficient than BVHs:
When taking movement over time into account, each object would need separate bounding volumes for separate segments of time,
forcing a ray to not just check one bounding volume, but multiple per object.
This also means that in order to be able to create meaningful bounding volume hierarchies,
each object's bounding volumes would need to be separated at the same points in time,
which can lead to redundancies if objects move at different times.
Calculating a useful BVH becomes impossible.
\newline
The amount of chunks, in turn, does not change:
They can be adapted simply by storing not just which objects are inside them,
but also when each object enters and exits the chunk.
If chunk contents are calculated correctly,
this means that no intersection checks take place for objects that aren't inside the given chunk at the given time.

\section{Data Structure}

In a simple system, a chunk stores a list where each entry represents an object inside it.
To accommodate for objects moving in and out of chunks, entries can instead contain three fields:
One containing the index of the object in question,
one containing the time at which the object enters the chunk
and one containing the time at which the object leaves the chunk.
Since the latter two fields might both be optional if the scene starts/ends with the object inside the scene,
% TODO source for sum/product type
this can be nicely represented using a sum type such as Rust's Enumerators or C's union types with different states:

\begin{verbatim}
// Object stays within chunk for the whole scene
// only store the index
Static(object)
// Object enters and exits chunk at the given times
Dynamic(object, time_entry, time_exit)
// Object enters chunk at the given time
// and stays until the end
Final(object, time_entry)
\end{verbatim}

As the scene's start time is known and the state of objects before it is irrelevant,
a state containing only an exit time is not necessary as it can be modelled using the
\verb|Dynamic| state with a \verb|time_entry| matching the scene's starting time.
Using a more common product type system, chunk entries can instead be represented as a
struct or class where the entry and exit times are optional or nullable fields.
\newline
A ray traversing this scene can now simply check when it enters and exits a chunk
and pick out the objects to check for intersections with accordingly.
When using sum types, the space requirements for static objects only increase by one byte denoting the type's variant
(with even that potentially getting left out, as Herzog showed~\cite{He23}).
For moving objects, only up to two additional fields plus the variant field are required,
with the timestamp fields' size depending on the implementation.
Compared to the performance gains from avoiding needless intersection checks,
this additional space requirement is comparatively minimal.

\section{Calculating Chunks}

\section{Traversing Chunks}
