\chapter{Evaluation}\label{ch:Evaluation}

In order to evaluate the effectiveness of the newly developed interpolating intersection logic,
three test cases will be used and compared to the snapshot method.
\newline
The first test case is simple: A receiver moves towards an emitter at 1/9th the speed of sound.
It starts 342.2 meters away from the emitter.
The emitter sends out a sine wave at 440Hz for 1 second.
No other objects are in the scene.
\newline
This is essentially the example described in \ref{im:SnapshotExplain}
and is used to demonstrate the basic differences between interpolated and snapshot methods.
\newline
% TODO image
The second test case takes place inside a rotating rectangular room.
Said room is 4 meters in width and length and 3 meters tall.
The receiver is in the very middle of the room,
with the emitter being 1.2 meters above it.
The room rotates around the Z-axis once per second.
\newline
This is used to test whether the differences between interpolated and snapshot methods
lead to a notable difference in a slightly more practical scenario.
\newline
% TODO image
The third test case takes place inside a rotating L-shaped room,
as denoted in IMAGE, with the receiver being at the origin
and the emitter being 0.5 meters above it.
The room again rotates around the Z-axis, but takes three seconds for a full rotation.
\newline
This case is used to demonstrate the shortcomings of the interpolating intersection logic
and the need for further research for realistic simulations.

\section{Approaching Receiver Scene}

In theory, two effects should be observable here.
\newline
Firstly, the ray would take 1 second to arrive at the emitter's position.
As the emitter is travelling towards the ray and covering a tenth of the distance in the time the ray travels the remainder,
the sound should already start at 0.9 seconds.
\newline
Secondly, due to the emitter's fast movement,
the doppler effect would lead to a shift in frequency.
\newline
Using the well-known doppler effect formula with a propagation speed \(c = 342.2 m/s\), an emitter speed \(v_s = 0 m/s\),
a receiver speed \(v_r = 342.2/9 m/s = 38.0\bar{2} m/s\) and a base frequency \(f_0 = 440 Hz\),
the resulting frequency can be calculated as
\begin{equation}\label{eq:Doppler}
    f = \frac{c + v_r}{c + v_s} f_0 = \frac{342.2 m/s + 38.0\bar{2} m/s}{342.2 m/s} \cdot 440Hz = 488.\bar{8} Hz
\end{equation}

% TODO image of sound waves
As expected, the first effect can be observed in the interpolated version of the simulation,
but not in the simulation using the snapshot method.
\newline
This is because for the snapshot method,
only the initial position of the receiver is relevant to the distance a ray needs to travel until it arrives at said receiver.
Thus, the ray has to travel the full 343.2 meters for the initial impulse response, as shown in \ref{im:SnapshotExplain}.
\newline
The interpolated version instead takes this effect into account correctly.
\newline
One notable detail is that the interpolated method also simulates the doppler effect more accurately.
The resulting signal from the interpolated method exactly matches the \(488.\bar{8} Hz\) calculated in \eqref{eq:Doppler},
while the snapshot method instead arrives at a frequency of approximately \(495 Hz\).
% TODO explanation
\newline
% TODO numbers
Performance wise, the simulations barely differ.
This is presumably because each impulse response is calculated using only a single ray
which only needs to check for intersections with a single object,
rendering the increased intersection check costs insignificant.
\newline
Additionally, the snapshot method implementation has the overhead of re-calculating
the chunks for each snapshot scene.
This is insignificant in simulations where many rays are used
as the chunk calculation cost is small by comparison,
but for this single-ray simulation, the overhead evens out the performance gained from having a cheaper intersection calculation.
\newline
A noteworthy implementation detail becomes apparent from the waveform resulting from the simulation:
As the input signal is a single sine wave and the doppler effect would only raise its frequency,
the resulting wave should still be a pure sine wave.
In contrast, the actual result for both simulation types features aliasing effects.
\newline
This is because the simulation is run as the same sample rate as the input signal.
For each input sample, one impulse response is calculated.
As the doppler effect speeds up the incoming signal,
but the sped up signal is not upsampled accordingly,
thus creating the same effect as if the signal was downsampled and stayed at the same frequency.
\newline
To alleviate this, one would need to either run the signal through a low-pass filter ahead of time
or have both the input signal and simulation at a higher sample rate,
then filter and downsample to the target sample rate afterwards.
\newline
The former works, but might damage a signal more complex than a sine wave
and requires knowledge of the scene ahead of time.
For a more complex scene where the downsampling effect cannot be easily calculated prematurely,
this approach becomes unusable.
\newline
The latter leads to increased computation cost as more impulse responses need to be calculated for more signals,
but works with any arbitrary scene.
