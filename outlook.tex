\chapter{Outlook and Further Research}\label{ch:Outlook}

As mentioned in \autoref{sec:Scope},
the research from this thesis alone is not yet sufficient to fully simulate rotating rooms
and other scenes involving rapid movement
as the physical forces created in those scenarios and their effect on the sound waves are disregarded.
\newline
In the case of rotating rooms in particular,
the centrifugal force created by the rotation would lead to pressure gradients within the room.
Exploring the exact effect those pressure gradients have on sound waves
and how to simulate this using geometric methods would require collaboration between physicists and computer scientists.
\newline
As a general outline, air pressure differences probably have an effect on the propagation speed and direction of sound waves.
With the rotating L-Shaped room used as one of the examples for this thesis,
this implies that sound waves are directed away from the rotating walls.
A ray tracer would thus presumably have to calculate the pressure gradients ahead of time,
then have the rays travel in curved rather than linear trajectories accordingly.
\newline
Since the ray equation used to derive the intersection check equations in \autoref{ch:Intersection} would thus change,
new equations would have to be derived using a different formula to model the ray according to the changed requirements.
The derivation can still follow the same scheme as the one shown in this thesis.
For surface intersection checks, only the last part of the derivation from~\eqref{SurfaceAfterCross} onward
(after already having calculated and substituted \(g_{0..2}\)) would diverge from the derivation in this thesis.
\newline
As the ray equation will then become more complicated than a simple linear equation,
the resulting equation may also become more complicated.
If, for example, the ray would now be modelled as a quadratic equation,
the polynomial derived for surface intersection checks would become a fourth degree one,
rather than the third degree one it is with the linear equation.
If the polynomial degree goes beyond four
or the resulting ray equation cannot be resolved to a polynomial to begin with,
solving the equations algebraically becomes impossible,
instead requiring numeric approximations of their roots.
This will in turn again highly increase the computing cost associated with intersection checks.
\newline
Another field for further research could be real-time applications.
The intersection logic developed for this thesis can only work for pre-calculated rooms
as all movements in the scene need to be known ahead of time,
rendering dynamic scenes such as ones developing in real time incompatible.
\newline
A dynamic approach could work by not calculating the rays' entire movements at emission time,
but instead keeping track of all rays moving through the scene at a current point
and incrementally continuing each ray's journey through the updating scene at recurring intervals.
\newline
Considering the higher cost for intersection checks,
the additional memory costs to keep track of rays
and the requirement for physics calculations for a proper realistic simulation as outlined above,
performance might also become a problem for real-time scenes
even when using a fast simulation method such as the AD-Frustum developed by Chandak et al.~\cite{Cha08},
and new optimisations will be needed.
