\chapter{Introduction}

\section{Motivation}

While there has been plenty of research into simulating sound propagation using both geometric and numerical methods,
nearly all of it only concerns itself with static scenes,
where neither objects within the scene nor the sound emitter nor receiver move with time.
\newline
Simulation of dynamic or moving scenes is mostly unexplored:
Raghuvanshi et al.~\cite{RS10} explore a numerical approach to handle dynamically moving emitters and receivers and
Chandak et al.~\cite{Cha08} attempt to simulate dynamic scenes in real-time,
with support for dynamically moving scenes but sacrificing accuracy.
Besides the AD-Frustum described in~\cite{Cha08} being prone to minor visibility issues,
it also fails to consider that as sound does not travel instantly,
objects and receivers can still move while a sound wave is traversing the scene.
This same issue also occurs in EAR, a simulation tool that supports blender's keyframe system for moving scenes.
\newline
% TODO: cite EAR
This is because of the approach both Chandak~\cite{Cha08} and EAR use, which will be called the snapshot method in this thesis:
At the time rays are emitted to create an impulse response, a snapshot of the scene in its current state is taken,
then rays are bounced through this snapshot.
This approach comes with a few advantages:
As the snapshot is essentially just a static scene, the same bouncing logic used for static scenes can be copied without changes.
Also, crucially, knowledge of how the scene will move over the time the ray spends bouncing around it is not required.
All data necessary to simulate the bouncing is available at the time the ray is emitted,
without a need for information on how the scene will continue to move.
This is especially helpful for real-time simulation of dynamic scenes, as data about how the scene will continue to move is not
fully known at runtime.
\newline
The downside of this snapshot approach is that it tends to introduce errors when objects or receivers move at high speeds.
% TODO visualisation
As a simple example case, take the scene described in (IMAGE):
A receiver starts 343 meters away from an emitter and moves towards it at 1/9th the speed of sound, roughly 38 meters per second
(137.2 kilometers per hour, a speed most modern cars can reach without problems).
Using the snapshot approach, a ray traveling directly from emitter to receiver would arrive after travelling the full 343 meters,
taking 1 second for it to arrive at the receiver.
In actuality, in the time the ray takes to travel the first 90\% of that distance,
the receiver has already travelled the remaining 10\%, making for a response time of 0.9 seconds rather than 1 second.
\newline
While Bilibashi et al.~\cite{BVD20} have attempted to solve this issue,
they only aimed to simulate waves bouncing between a few set points, namely cars,
rather than simulating full room acoustics.

\section{Scope}

This thesis proposes a method to simulate rays bouncing through arbitrary scenes with moving receivers and/or objects,
assuming all movement within the scene is known at time of calculation.
An improved way of checking for intersections between rays and objects is developed, accommodating for this new requirement.
Additionally, a method is developed to to losslessly and efficiently store the multiple impulse responses created by re-calculating
the impulse responses for different points in time.
The goal of this research is to simulate effects such as the situation described in (IMAGE) without errors introduced by the snapshot method
as well as accurately recreate the acoustics of a hypothetical, rapidly rotating room.
Three test cases are developed for this and compared to an implementation of the snapshot method: 
An empty scene with the sound receiver approaching the sound emitter at 1/3 the speed of sound,
a square room rapidly rotating
and a large, L-shaped room also rapidly rotating around one of its ends, with the receiver and emitter both sitting in said end.
\newline
Side effects of moving scenes, such as sounds emitted by moving objects, are discarded as they are irrelevant to
the changed intersection logic.
A note-worthy side effect that gets ignored is mass inertia:
The example case where this would become relevant is the inside of a linearly moving enclosed room, such as a driving car.
Due to inertia, the acoustics inside this moving room are the same as if the car stood still.
Since this effect is only relevant in a niche scenario and it can be simulated using a method that ignores movement entirely,
it can be ignored for this research.
\newline
Real-time applications cannot use this proposed method as it requires knowledge of the scene's further development ahead of time.
Further research is required to develop an alternative method for real-time or dynamic simulations.